\documentclass{article}
\usepackage{typearea}  % package “geometry” can’t change paper mid-document
\usepackage{fontspec}
\usepackage{graphicx}
\usepackage{verbatim}  % for \begin{comment}
\usepackage{tikz}
\usepackage{expl3}

\usetikzlibrary{calc}
\usetikzlibrary{shapes.geometric}
\usetikzlibrary{quotes}

\pagestyle{empty}  % no page numbers please
\setmainfont{Charis SIL}
\newfontfamily\symbolfont{Dyuthi}

\KOMAoptions{paper=a0,paper=landscape}
\recalctypearea

\begin{document}

\ExplSyntaxOn
\dim_new:N \gDinAZeroLongDim
\dim_new:N \gDinAZeroShortDim
\dim_gset_eq:NN \gDinAZeroLongDim \pdfpagewidth
\dim_gset_eq:NN \gDinAZeroShortDim \pdfpageheight
\ExplSyntaxOff

\newcommand{\symbolfontsettings}{\symbolfont}

\tikzset{
    every picture/.style={
        outer sep=auto,
    },
    every pic/.style={
        draw,thick,font=\symbolfontsettings,
    },
    symbol label/.style={font=\tiny,below,inner sep=1pt},
    health symbol/.pic={  % circle
        \begin{scope} [color=red!75!black,]
            \draw (0,0) circle [radius=2mm];
            \node[symbol label] at (0,-2mm) {\siemerpiclabel};
            \node {\tikzpictext};
        \end{scope}
    },
    stamina symbol/.pic={  % square
        \begin{scope} [color=blue!75!black,]
            \draw (-135:2.5mm) coordinate (A) rectangle (45:2.5mm) coordinate (B);
            \path (A) -| (B) node[pos=.25,symbol label] {\siemerpiclabel};
            \node {\tikzpictext};
        \end{scope}
    },
    deadeye symbol/.pic={  % triangle
        \begin{scope} [color=green!50!black,]
            \draw (90:2.5mm) -- (210:2.5mm) -- 
                node[midway,symbol label] {\siemerpiclabel} (330:2.5mm) -- cycle;
            \node {\tikzpictext};
        \end{scope}
    },
    siemer pic label/.store in=\siemerpiclabel,
}

\ExplSyntaxOn

\cs_new_protected:Nn \generateSymbolMacro:Nnnnn {   % \ginsengAlaskan {health} {G} {ginsengAlaskan} {alaskan}
    \cs_new_protected:Npn #1 ##1 {  % generated command takes a coordinate list
        \tikzset{
                this/.style={
                    "{#3}",siemer~pic~label,
                }
        }
        \tl_if_eq:nnF {#4} {#5} {  % the caller uses \foreach and will feed #4==#5 if there is no label
            \tikzset{
                this/.append~style={siemer~pic~label={#5}},
            }
        }
        \pic foreach \pos in {##1}
            [this,at=\pos] {#2~symbol};
    }
}

\cs_generate_variant:Nn \generateSymbolMacro:Nnnnn { coooo }

\foreach \siemer_style / \siemer_symbolLetter / \siemer_command / \siemer_label in {
    health/G/ginseng,
    health/G/ginsengAlaskan/alaskan,
    health/G/ginsengAmerican/american,
    health/Y/yarrow,
    health/E/englishMace,
    health/M/milkWeed,
    stamina/B/burdockRoot,
    stamina/S/sage,
    stamina/S/sageDesert/desert,
    stamina/S/sageHummingbird/hummingbird,
    stamina/S/sageRed/red,
    stamina/V/violetSnowdrop,
    stamina/W/wildFeverfew,
    deadeye/I/indianTobacco,
    deadeye/C/currant,
    deadeye/C/currantBlack/black,
    deadeye/C/currantGolden/golden,
    deadeye/P/prairiePoppy
} {
    \exp_args:No \generateSymbolMacro:coooo 
        \siemer_command \siemer_style \siemer_symbolLetter \siemer_command \siemer_label
}

% How to replace the background bitmap map?
% 1. compile the original with the grid enabled and the full map on some A0 page in the document
% 2. check where the grid’s origin (coordinate (0,0)) and Annesburg’s train station dot (around (90,50.1)) are
% 3. replace the image, set originShift to (0,0) and fullMapScale to 1
% 4. if the image is satisfactory positioned in A0, note the coordinate of what the origin should be
% 5. set the originShift to noted coordinate and check the result
% 6. fiddle with fullMapScale; use values around 1, until Annesburg’s train station is on (90,y) → (90,50.1)

\tl_new:N \fullMapScale
\tl_new:N \originShift
\tl_new:N \originCorrection
\tl_gset:Nn \originShift {(10.0,12.5)}  % where the New Austin border hits the water
\tl_gset:Nn \fullMapScale {0.9969}   % (90,y) should go through the train station dot in Annesburg ~(90,50.1)
\tl_gset:Nx \originCorrection {($-1*\originShift$)}  % for use in crop pages

\ExplSyntaxOff
\newsavebox\fullmap
\sbox\fullmap{  % this comment or expl3 syntax is required to keep this picture savebox snug
    \begin{tikzpicture}
        \begin{scope}  % no `overlay`; this bitmap should provide the bounding box of this savebox
            \node[above right,inner sep=0] {
                \includegraphics[
                    width=\gDinAZeroLongDim,
                    totalheight=\gDinAZeroShortDim,
                    keepaspectratio,
                ] {rdr2/Red-Dead-Redemption-2-Full-World-Map.jpg}
            };
        \end{scope}
        \begin{scope}[
                overlay,  % the background map bitmap provides the bounding box—the rest should not
                shift={\originShift},
                scale=\fullMapScale,
                ]
            \draw[help lines] (-10,-20) grid (120,100);
            \node foreach \x in {-10,-5,...,120} foreach \y in {-20,-15,...,100}
                [node font=\fontsize{4}{5}\selectfont,above right]
                at (\x,\y) {\x,\y};

            % New Austin
                % Gaptooth Ridge
                    \sage{
                        (4.5,1.5),
                        (8.5,1.5),
                        (6.3,10),
                        (7.9,13.4),
                        (11.2,14.6)
                    }
                    \sageDesert{
                        (4.5,5.5)
                    }
                    \currant{
                        (8.1,3.9),
                        (9.5,5.5),
                        (9.2,7.1),
                        (11.3,7.1),
                        (7.4,10.7),
                        (8.1,11.7),
                        (11.2,15.4),
                        (9.8,14.2)
                    }
                    \currantBlack{
                        (9.5,10.7)
                    }
                % Cholla Springs
                    \wildFeverfew{
                        (14,14)
                    }
                    \sageRed{
                        (16,16)
                    }
                    \currant{
                        (18,16)
                    }
                % Rio Bravo
                    \englishMace {
                        (8,20)
                    }
                % Hennigan’s Stead
            % West Elizabeth
                % Tall Trees
                % Great Plains
                % Big Valley
            % Ambarino
                % Grizzlies West
                % Grizzlies East
            % New Hanover
                % The Heartlands
                % Cumberland Forest
                % Roanoke Ridge
            % Lemoyne
                % Scarlett Meadows
                % Bluewater Marsh
                % Bayou Nwa
        \end{scope}
    \end{tikzpicture}  % this comment or expl3 syntax is required to keep this picture savebox snug
}
\ExplSyntaxOn

\tl_new:N \l_pictureShift_tl
\cs_new_protected:Nn \cropPage:nnnn { 
    % {A4}, {landscape}, {(20,10)}, {1}
    % zoom factor
    \KOMAoptions{paper=#1, paper=#2}
    \recalctypearea

    \tl_set:Nx \l_pictureShift_tl {($-\fullMapScale*#3$)}
    \begin{tikzpicture}[
            remember~picture,
            overlay,
            shift=(current~page.south~west)
            ]
        \node[
                above~right,
                inner~sep=0,
                scale=\tl_if_empty:nTF {#4} {1} {#4},
                shift=\tl_if_empty:nTF {#3} {(0,0)} {\originCorrection},
                shift=\tl_if_empty:nTF {#3} {(0,0)} {\l_pictureShift_tl},
                ] {\usebox\fullmap};
    \end{tikzpicture}
}

\seq_new:N \l_countyData_seq
\seq_set_split:Nnn \l_countyData_seq {;} {
    p/(-3.5,-5);  % Gaptooth Ridge
    l/(10.7,2.2);  % Cholla Springs
    l/(6,-5);  % Rio Bravo
    l/(27.8,3.8);  % Hennigan’s Stead
    l/(33.1,15.8);  % Tall Trees and Great Plains
    l/(32.7,32.3);  % Big Valley (south part on Tall Trees and Great Plains)
    p/(40.3,43.8);  % Grizzlies West (eastern part on Grizzlies East, southern tip on Big Valley)
    l/(57.1,44);  % Cumberland Forest and Grizzlies East
    l/(55,30);  % The Heartlands (north east tip on xxx; east tips on Big Valley)
    p/(76.5,36.5);  % Roanoke Ridge (south tip on Bluewater Marsh)
    p/(63.7,14.5);  % Scarlett Meadows
    p/(73.5,17.5);  % Bayou Nwa and Bluewater Marsh  # add (2,0) for more Sisika
}

\cs_generate_variant:Nn \cropPage:nnnn { nffn }
\seq_new:N \l_orientationPosition_seq
\cs_new_protected:Nn \cropPages:Nnn {  % \l_countyData_seq {a4} {1.2}
    \seq_remove_all:Nn #1 {}  % “support” trailing delimiter on input seq
    \seq_map_inline:Nn #1 {
        \seq_set_split:Nnn \l_orientationPosition_seq {/} {##1}
        \cropPage:nffn
            {#2}
            {\str_if_eq_x:nnTF {\seq_item:Nn \l_orientationPosition_seq {1}} {l} {landscape} {portrait}}
            {\seq_item:Nn \l_orientationPosition_seq {2}}
            {#3}
    }
}

\cropPage:nnnn {a0} {landscape} {} {}  % with empty arguments it is actually the full map
\cropPages:Nnn \l_countyData_seq {a4} {1.2}
\cropPage:nnnn {a4} {landscape} {(-1,-5)} {.575}  % New Austin
\cropPage:nnnn {a4} {landscape} {(31,15.5)} {.45}  % West Elizabeth, Ambarino, New Hanover Lemoyne

\end{document}